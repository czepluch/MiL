\section{Business Model} % (fold)
\label{sec:business_model}
\subsection{Customer Segments} % (fold)
\label{sub:customer_segments}
\begin{itemize}
	\item \textbf{Music enthusiasts}\\
			This customer segment is interested in experiencing live music, and discovering new artists. \ref{i2q2} \ref{i2q4}
	\item \textbf{Families with children}\\
			This customer segment wants cultural offers which include activities for the entire family, including children. \ref{i2q2}
	\item \textbf{Local populous}\\
			This customer segment is interested in strengthening the local community, for example through cultural events. \ref{i2q6}
	\item \textbf{Party people}\\
			These customers want to party. They are interested in affordable prices on alcohol, and offers of specific genres of party music. \ref{i2q5}
	\item \textbf{Sponsors}\\
			This segment is interested in branding itself as supporting both local cultural events, and danish music culture. \ref{i1q7}
	\item \textbf{Service providers}\\
		This segment is interested in using the festival as a platform for offering their goods and services to the audience of the festival. For example, this could be caterers interested in selling food at the festival \ref{i1q6}
\end{itemize}
\subsection{Value Propositions} % (fold)
\label{sub:value_propositions}
\begin{itemize}
	\item \textbf{Strong musical profile}\\
	 Musik i Lejet strives to maintain a well-defined and contemporary musical profile. This allows music enthusiasts to experience both popular artists and discover new music. \ref{i2q1} \ref{i2q2}
	\item \textbf{Free entrance for children and elderly}\\
	Entrance to the festival is free for children and the elderly. This makes the prospect of making the festival a family event more affordable. \ref{i2q2}
	\item \textbf{Designated children's stage}\\
	Each day, a number of artists performing children's music at a designated stage. This as desirable for families with children, as it makes the festival an event for the entire family. \ref{i2q2}
	\item \textbf{Nightly after-party}\\
	Each night an after-party is held on the beach, with DJs and an open bar. This is desirable to anyone interested in attending wild parties. \ref{i2q5}
	\item \textbf{Unique location}\\
	Musik i Lejet is held at the beach of Tisvildeleje. This adds to a unique festival atmosphere, and contributes to a strong brand for Musik i Lejet. This is valuable for anyone wanting to associate with the festival, and use the festivals image to brand themselves. Its also of interest to most of the customers who attend for the festival experience. \ref{i2q7}
	\item \textbf{History of sold out tickets}\\
	The festival is increasingly popular, and boasts a history of sold out tickets. This is assures sponsors of getting the desired exposure by supporting the festival with cash donations or goods and services. In addition, its attractive to service providers interested in selling services at the festival. \ref{i1q3}
	\item \textbf{Frequent coverage in media}\\
	Musik i Lejet is often mentioned in music and mainstream media. This is useful for sponsors, as a strong brand for festival reflects well on their involvement with it. \footnote{http://gaffa.dk/nyhed/78769} \footnote{http://politiken.dk/ibyen/nyheder/musik/ECE2005317/sommerferie-festival-samler-staerke-danske-navne-paa-stranden/}
	\footnote{http://soundvenue.com/musik/2013/02/musik-i-lejet-spiller-med-musklerne-46513}
\end{itemize}
\subsection{Channels} % (fold)
\label{sub:channels}
\begin{itemize}
	\item \textbf{musikilejet.dk}\\
		The festival uses the page musikilejet.dk, to provide information about the festival program, ticket sales (by a link to billetto.dk), and general information about the festival. The page also provides contact information on board members and arrangers in order for customers to be able to contact the relevant people int the organization.
		\footnote{http://musikilejet.dk/}
	\item \textbf{Local advertisement}\\
		Local advertisement is used to raise awareness of Musik i Lejet's services. Posters and brochures available at local businesses display information about the festival program and dates. \ref{i2q6}
	\item \textbf{Music media}\\
		Musik i Lejet offers accreditation to critics and journalists who visit the festival with the purpose of reporting on concerts and events. This means Musik i Lejet is often covered in magazines and newspapes such as Gaffa, Soundvenue, Politiken ect. \footnote{http://gaffa.dk/nyhed/78769}  \footnote{http://politiken.dk/ibyen/nyheder/musik/ECE2005317/sommerferie-festival-samler-staerke-danske-navne-paa-stranden/}
		\footnote{http://soundvenue.com/musik/2013/02/musik-i-lejet-spiller-med-musklerne-46513}
	\item \textbf{Social Media}\\
		Musik i Lejet extensively uses social media to promote the festival, in addition to receiving feedback from and communicating with customers. \footnote{https://www.facebook.com/musikilejet}
	\item \textbf{Billetto.dk}\\
		Musik i Lejet offers tickets in advance sale through billetto.dk \footnote{http://billetto.dk/musikilejet}
	\item \textbf{The festival site}\\
		The festival site is where all events and services are offered to customers. Tickets not sold in advance can be purchased at the festival site.
\end{itemize}
\subsection{Customer Relationships} % (fold)
\label{sub:customer_relationships}
Access to ticket purchase is done primarily through self-service at billetto.dk. Throughout the year, maintenance of customer relationships is done primarily through social media. Musik i Lejet has a dedicated communication department, which answers posts made to the festivals Facebook wall, and comments made on Musik i Lejet posts on Facebook. \footnote{http://musikilejet.dk}

\subsection{Revenue Streams} % (fold)
\label{sub:revenue_streams}
\begin{itemize}
	\item \textbf{Grants and donations}\\
	Musik i Lejet recieves financial support through various local and national government grants, as well as from private foundations. in addition Musik i Lejet is sponsored by various private companies with cash donations and/or goods and services. \ref{i1q7}
	\item \textbf{Festival tickets}\\
	Tickets to Musik i Lejet are purchased as a one time payment. Prices for both par-tout tickets, and one day tickets are fixed. In addition, its possible to buy a ticket which includes housing at a local vacation resort. \footnote{http://billetto.dk/musikilejet}
	\item \textbf{Bar sales}\\
	Bars at the festival sell both alcohol and soft drinks at fixed prices. \ref{i1q8}
	\item \textbf{Stalls}\\
	Business owners who wish to sell food can buy stall space at the festival at a fixed price. \ref{i2q6}
\end{itemize}

\subsection{Key Resources} % (fold)
\label{sub:key_resources}
\begin{itemize}
	\item \textbf{The festival site}\\
	The area used as festival site is essential for Musik i Lejet. The space is used for free, being lent by the local municipality. \ref{i1q7}
	\item \textbf{Construction materials, toilet facilities and decoration}\\ 
	A large quantity of construction materials for stalls, bars, signs and more is necessary for building the festival site. Moreover fencing and toilet facilities is rented from an external supplier. Lastly, decoration such as colored lanterns and paint is necessary for creating the festivals visual identity at the festival site. \ref{i1q7}
	\item \textbf{Stage and lighting}\\
	Stage and lighting for the concerts at the festival is rented from a supplier. \ref{i1q7}
	\item \textbf{Volunteers}\\
	Volunteers are necessary for virtually every aspect of the festival operation. This includes selling alcohol and soft drinks in the bars, clean up before opening on each of the festival, band care and more. For Musik i Lejet 2013, approximately 500 volunteers were working in the course of 3 days. \ref{ws1}
	\item \textbf{Musicians}\\
	A crucial resource of the festival is naturally the bands and artists who perform. This includes usage of their intellectual property. Musik i Lejet has a fixed price deal with KODA, the danish copyright management society.
	\item \textbf{Stage technicians}\\
	Stage technicians are necessary for building and operating stage sound and lighting. Technicians paid through the supplier of stage equipment. \ref{i1q8}
	\item \textbf{Arrangers}\\
	Musik i Lejet relies on the volunteer work of the 20-30 people who plan and manage festival operations. \ref{i1q6}
	\item \textbf{Food and drink}\\
	Food sales at the festival is managed by service providers who buy stall space at the festival site. Alcohol and soft drinks is provided by Carlsberg, one of the main sponsors of the festival, at an affordable price. Machinery for draft beer is lent by Carlsberg, in accordance with a sponsorship agreement. \ref{i1q7}
	\item \textbf{Security}\\
	Bouncers are hired through a agency. \ref{i1q8}
	\item \textbf{Donations and sponsorships}\\
	Musik i Lejet is highly dependent on the financial resources obtained through sponsorship agreement and grant funds from public and private foundations. \ref{i1q8}
\end{itemize}

\subsection{Key Activities} % (fold)
\label{sub:key_activities}
(baseret på organisations diagram og møde med andreas)
\begin{itemize}
	\item \textbf{Strategic planning}\\
	A big part of the work of Musik i Lejet is to make strategic decisions about the practical framework of the festival, such as festival days and opening hours, and branding decisions such as guidelines for the musical profile and visual identity of the festival. These decisions are generally made by the board. \ref{i2q8}
	\item \textbf{Operations planning}\\
	The majority of the time of the arrangers of Musik i Lejet is spent planning the operations of the festival. The arrangers have separate areas of responsibility, but dependencies between the different groups of arrangers exist. As such, the arrangers need to coordinate their actions and share knowledge about decisions made that affect other arrangers. The primary areas of responsibility within the group of arrangers are:
	\begin{itemize}
		\item Communication
		\item Music booking and backstage care
		\item Tech (electricity, stalls, stage etc.)
		\item Volunteers (recruiting, shift planning etc.)
		\item Bar
		\item Authority (permits, cooperation with municipality etc.)
		\item Economy and Legal
		\item Food sales
		\item Safety
		\item Fund-raising
		\item Festival site construction
		\item Logistics
	\end{itemize}
\end{itemize} \ref{org_chart}

\subsection{Key Partnerships} % (fold)
\label{sub:key_partnerships}
(baseret på møde med andreas)
\begin{itemize}
	\item \textbf{Municipality}\\
	The festival site is used with the permission of the municipality, which is a natural prerequisite for the execution of the festival. As such, the relationship with the municipality is vital for the continued success of the festival. It is the job of Musik i Lejet to convince local authorities, that the festival generates cultural value for the local population, as well as monetary value for the municipality  by attracting tourist business. \ref{i1q7}
	\item \textbf{Police and fire department}\\
	During the festival, access to and from the festival by the main access road is effectively blocked by traffic generated by the festival. This means that alternative routes for the local police force and fire department must be planned and coordinated with these authorities in advance. \
	\item \textbf{Sponsors}\\
	As mentioned, sponsors are vitally important for the success and indeed survival of the festival. Maintaining the relationship with shot-callers at sponsoring cooperations and government and private grant foundations is critical. \ref{i1q7}
	\item \textbf{Booking agencies}\\
	The ability to secure affordable prices on performing artists is crucial to the success of the festival. As such, the partnership with prominent booking agencies is crucial.
\end{itemize}

\subsection{Cost Structure} % (fold)
\label{sub:cost_structure}
(baseret på møde med andreas og møde med kristian og stakkeman)
Musik i Lejet has a relatively small budget for each festival (about 1.7 million dkr), so this naturally makes the cost structure of the festival cost driven. It is vitally important to obtain good offers on every resource necessary for executing the festival. Moreover, one of the main objectives of Musik i Lejet is to maximize diversity of the festivals audience. One of the ways of doing this is to keep ticket prices low. \ref{i1q8} \ref{i2q1} \ref{i2q2}