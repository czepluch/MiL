\section{Environment} % (fold)
\subsection{Market forces}

\subsubsection{Market segments}

MiL seeks to welcome an as broad span of segments as possible. This is part of the vision of delivering a product that versatile music festival for everybody, purposely avoiding exclusion of customer segments. With this mentioned the event of MiL has its main targets, which are the following (notice also mentioned in the business model canvas):
\\ \\
- Music enthusiasts
MiL takes its musical profile serious and is also heavily incorporating in the identity of the festival.
Pleasing this segment strengthens the profile helping maintain the identity and niché that MiL seeks. As enthusiasts are likely to add to contribute awareness in the shared industry, attracting and pleasing these MiL creates self-reinforcing publicity effects.
\\ \\
- The “party people”/”festival enthusiast”
A well known segment of customers heavily attracted by pop-tendencies with a great contribution to revenue. The reason of this segments importance lies in its potential growth potential, which in comparison to the other  targeted market segments (fully listed in the business model canvas) has shown explosive potential. 
\\ \\
The main threat for MiL lie in keeping pace in respect to the size of the festival, as well as the 2 previously mentioned segments. As well as being the most important customer segments, they are likewise the most vulnerable.

\subsubsection{Switching costs}
...
(values binding the customer to MiL)
\begin{itemize}
	\item - Price
	\item - Musical profile
	\item - Identity
	\item - Accomody possibilities
	\item - Location
\end{itemize} 

(values if altered could inflict defection)
\begin{itemize}
	\item - Ticketprice
\end{itemize}

Musik i Lejet produces a unique product and which provides its niché, is its location and strong music profile, accompanied with a low price. Because of this its hard to purchase other offers with the same value. This combination is the sales point which exposes the event to expectations of a low price / high value product, meaning altering of this profile threatens defecting customers. 

\subsection{Industri Forces}
\subsubsection{Competitors}

The industry of danish music festivals is a growing market, attracting more people each year and has shown indications of a trend towards increased appreciation of larger live events. The range of festivals of the market spans over a wide range of types of events, from heavy rock festivals like Copenhell with a guest count of 11.000, to the multigenre Roskilde festival with a size of 100.000 guests. These different festivals differentiate significantly in vision, identity, and as mentioned, size which leads to fragmentation of customers, producing customer segments with different needs. This means that when considering the market, competition does not exist between all the industry organisations, but rather “locally” between festivals that target the same customer segments. 
\\ \\
With MiLs cost/value/length/music profile, direct competing festival alternatives are significantly few in numbers, suggesting: Spot festival, Wonderfestiwall, and Trailerpark festival. 
\\ \\
Due to little amount of direct alternatives, the matter of substitute competition don’t oppose an immediate threat. This statement does on the other hand only hold as long as MiL can provide a product with the current price/value level. A increase in ticket price of 50 percent would equate MiLs offer, ultimately enabling customers to question the value of the MiL events unique location and atmosphere.

\subsubsection{Suppliers}
MiL has a strong bond to its suppliers through its cooperation with the local community. Materials, food stands and most other components of MiL are acquired through the local businesses and entities which has provided MiL a great deal of trust from their sponsors. With 5 years of success, and ability of being sold out on tickets MiL can provide security and trust in both directions between MiL and its suppliers and stakeholders.