\section{Environment} % (fold)
\subsection{Market forces}

\begin{itemize}
	\item \textbf{Party people}\\
	financially speaking, this is the most important segment that Musik i Lejet has. Sales of food and beverages made at the nightly after parties greatly exceeds what is sold at the rest of the festival (ref interview med Kristian og Stakke). The segment is attractive because of their willingness to spend money at the festival, but securing the segment may conflict with appealing to other segments such as families and elderly. The segment demands an offer contemporary, hyped party music, and low prices on and decent selection of alcoholic beverages.(reference til kristian og stakke)
	\item \textbf{Music enthusiasts}\\
	This segment is attractive because of their loyalty towards specific genres or bands. Its this segment that buys their ticket in advance, which contributes to financial stability for the festival (ref kristian og stakke). They demand an interesting and contemporary offer of live musical acts. This segment may be further broken down into segments of music enthusiasts within separate musical genres. One of the challenges of securing this segment is evaluating the trade-offs of weighting one genre over another in the booked acts.
	\item \textbf{Sponsors}\\
	The attractiveness of sponsors is that their presence reduces economic risk taken on by Musik i Lejet. They need festivals that will strengthen their own brand by association.
\end{itemize}

\subsubsection{Switching costs}
The following factors bind customers to festivals:
\begin{itemize}
	\item \textbf{Price}\\
	Festivals compete in terms of price, and is especially relevant to students or other low income segments.
	\item \textbf{Musical profile}\\
	Customers return to festivals with a well defined and consistent musical profile, if that profile appeal to their preferences.
	\item \textbf{Identity}\\
	Some festivals have reputations of being hip, some of being casual, some being something third entirely. Customers use festivals as a way of expressing their own identity (kristian og stakke)
	\item \textbf{Accommodation}\\
	Families with children and elderly prioritize appropriate accommodation. Changes to the accommodation available might cause the segments to use offers from competitors.
\end{itemize}
\subsection{Industri Forces}
\Subsection{Competitors}\\
Musik i Lejet simultaneously competes and cooperates with festivals of similar size and profile. Cooperation takes place through sharing of knowledge and experiences, as well as through free admission of Musik i Lejet to other festivals such as SPOT festival, Wonderfestiwall and Trailerpark festival, with the purpose of scouting acts and observing processes.

These festivals are also their competitors, as they are similar in prize range, audience size and musical profile. They will be discussed in the following (note that the list is by no means exhaustive):
\begin{itemize}
	\item \textbf{SPOT festival}\\
	SPOT festival competes by maintaining an image of being an industry festival. Acts are booked to spike the interest of record labels and other parts of the music industry, both nationally and internationally. This also makes the festival interesting to upcoming bands, which gives SPOT festival a unique opportunity to procure some of the most hyped acts. The interest from industry and artists draws the attention of music enthusiasts, who are interested in discovering novel contemporary music.(første interview andreas).
	\item \textbf{Wonderfestiwall}\\
	This festival competes by maintaining what can be described as a folksy musical profile. This attracts a certain kind of audience, who want to experience live performances by artists that they know in advance.
	\item \textbf{Trailerpark festival}\\
	Trailerpark can be considered one of the closest competitors to Musik i Lejet, as the market segments they address are similar: the music profile is similar, and the festival is geographically relatively close to Musik i Lejet.
\end{itemize}


\subsubsection{Suppliers}
Musik i Lejet is dependent on the following suppliers in the following ways:
\\begin{itemize}
	\item \textbf{Music acts}\\
	The quality of the music being preformed at Musik i Lejet influences customers view of the festival itself. (ref ??)
	\item \textbf{Service providers}\\
	Caterers and other service providers at the festival are both customers and suppliers. Musik i Lejet sells stall space to, which make them customers, but they also sell a product to the other customers at the festival. The quality of these products also affect the amount of value being delivered to customers.
	\item \textbf{Sound tech and technical personnel}\\
	
\end{itemize}

MiL has a strong bond to its suppliers through its cooperation with the local community. Materials, food stands and most other components of MiL are acquired through the local businesses and entities which has provided MiL a great deal of trust from their sponsors. With 5 years of success, and ability of being sold out on tickets MiL can provide security and trust in both directions between MiL and its suppliers and stakeholders.