\section{Podio Analysis}
\label{podio}

The following is an analysis of the way Musik i Lejet uses the web platform Podio. This is done as a summary of multiple observation rounds, where communication, decision making and information sharing on the media, has been documented. This analysis tries to draw the lines of the most important trends and practises. \\ \\

\subsection{Structure}
\label{podio_struc}
The structure of Podio for Musik i Lejet, very much mirrors the structure of the organisation. Every team has their own workspace, including a activity stream for status messages and questions, share files and links, and the ability to install Podio apps to further expand on the functionality. Along with this, an ``Emloyee network'' group is open for sharing material between the teams. Activity can be seen on Podio at most time of day, as people work when they have spare time, and can still communicate throughout the day\\ \\

\subsection{Communication and discussions}
\label{podio_com}
Podio is the official platform for communication internally in Musik i Lejet. Communication consists of different uses of status messages in the activity streams. Some of the most frequent uses of the stream, is to ask questions or report back after finishing a task and sharing information as seen in \ref{podio_1}. The questions are then answered in the comments, and additional information on a subject can be supplied if forgotten in the original message. There is no way of differentiating between a message with information or a question that have been asked, and a message could even incorporate both. \\

Generally answers to messages are quick, and keeps with the 24-hour rule, but it happens that messages are never answered as seen in \ref{podio_2}. This could be because a question has been answered on another media or similar. Though all communication should be conducted on Podio, often other forms of communication are used, because responsiveness can be better, or that a larger topic is easier discussed over the phone as seen in \ref{podio_3}. This can sometimes lead to people not being able to find information on Podio asseen in \ref{podio_4}\\

When a message gets four or more comments, the comment-section is collapsed to only show the two latest comment. If there is no comments on a message in the stream, it will sink down, and at some point not show up on the first page of the stream any more. This makes the information within the message and comments harder to find. In \ref{podio_5} additional information is given in the comments, but was not visible when scrolling through the stream\\

\subsection{Subjects}
\label{podio_subj}
Generally the communication between the arrangers are light-hearted, and this is also true for Podio, with social statuses with no relation to the planning of the festival as seen in \ref{podio_6}, and a very joking tone in the comments. The comment often gets sidetracked.

Many of the messages and questions revolve around people needing information about decisions; What is the status? Has it been made yet? If so, what was the result? If not, what is the deadline?. These questions mostly arise because the information is not available in Podio, or because it is buried in one of the activity streams. In \ref{podio_7} the question and comments show an exchange of information about deadlines. The are clearly set, but not shared on Podio.

\subsection{Decisions}
\label{podio_decis}
If an important decision has been made, and has some work is related to it, a task can be made with the Podio app. This can be decisions from meetings or decisions made ad-hoc. This is not always the case though. 

If a decision is made in the comments, a responsible arranger is often chosen or someone takes the responsibility upon themselves. Depending on the size of the decision, it can be documented in some way, maybe with a task, but mostly work will continue from here. The discussion in \ref{podio_7} continues to decide on more specific deadlines, but not making more task for this. 

\subsection{Documentation}
\label{podio_doc}
Much of the documentation on Podio is done through the different apps. Most used is the Meeting app and the Task app. The first is used for nearly all scheduled meetings, documenting attendants and the decisions made. The detail put into the minutes vary. The Task app is used extensively along with the meeting minutes, as you can make them directly from the Meeting app, coupling it with the information in the minutes. This is also where it is mostly used, but some also make tasks on their own. \\

Tasks and their descriptions vary greatly in size. Some confusion can arise, when arrangers do not agree on what is included in a specific task, and who has the responsibility. This mostly happens when i finishing point for a task is not defined in the description, or if the description is vague as seen in \ref{podio_8}.

Other than this, most documentation is kept in  different documents uploaded to Podio, such as contracts.

A specific kind of documentation not really present on Podio, is instructions and guidelines on how to use it. Some instructions are written when arrangers ask for help, but as with other information it sinks down out of the activity stream, when written in a status as seen in \ref{podio_9}. 

\subsection{Podio screenshots}
\label{podio_screens}

\subsubsection{1}
\label{podio_1}
\includegraphics[scale=0.7]{Pictures/Podio_1.png}

\subsubsection{2}
\label{podio_2}
\includegraphics[scale=0.7]{Pictures/Podio_2.png}

\subsubsection{3}
\label{podio_3}
\includegraphics[scale=0.7]{Pictures/Podio_3.png}

\subsubsection{4}
\label{podio_4}
\includegraphics[scale=0.7]{Pictures/Podio_4.png}

\subsubsection{5}
\label{podio_5}
\includegraphics[scale=0.7]{Pictures/Podio_5.png}

\subsubsection{6}
\label{podio_6}
\includegraphics[scale=0.7]{Pictures/Podio_6.png}

\subsubsection{7}
\label{podio_7}
\includegraphics[scale=0.7]{Pictures/Podio_7.png}

\subsubsection{8}
\label{podio_8}
\includegraphics[scale=0.7]{Pictures/Podio_8.png}

\subsubsection{9}
\label{podio_9}
\includegraphics[scale=0.7]{Pictures/Podio_9.png}

