%!TEX root = /Users/Abj/git/MiL/report/report.tex
\part{In-depth Analysis}
\label{prt:in_depth_analysis}
\section{Organizational setup, key players and KPI's}
The key players, concerned with this project, are the ones that have the highest level of interest, and the abilities to perform the needed changes towards its goal, such that the different levels of KPI's can be met. The different sections of the organisations described in the work domains \ref{sec:work_domains}, have different impacts on the realisation of deadlines, as expressed by the mapping in \ref{sec:stakeholder_map}. \\

The board has great interest in the status of task and their deadlines, as this is a tool for them to get an overview of the progress of the planning of the festival. This overview is essential to the board, as it enables them to track that the strategies are being kept, and that the different teams are keeping their budgets. This insures that the overall budget is kept, making ground for the financial stability that is a clear goal of the board (kilde: Andreas \& Kristian+Stakke), so that they can keep their current value propitiations \\

With the importance of the budgets, the team leaders gets a lot of power, as they are generally the ones responsible for their own teams budgets, and they are the ones that have the power to keep it. This comes down to, the team leaders being the primary owners of the different deadlines, both internally in the team, but also the ones agreed upon between teams, also making them one of the parties most interested in keeping the deadlines. \\

The other arrangers in the different teams, do not have as much interest in keeping the deadlines, as this is not their responsibility. They do however have nearly as much power to realise the deadlines by completing tasks, but lack the cross-team influence. \\


\section{Current work practices}
% Den måde de arbejder på lige nu, fører til problemet omkring deadlines opstår. Altså årsagen til deadlines ikke bliver overholdt og hvilke konsekvenser dette har i forhold til de nævnte KPI'er. \\
% Podio analysis, say/do, arbejds process for Stine F.\\
% Det her er aftalen for hvordan der skal arbejdes.\\
% Her er bevis for at de IKKE arbejder på den måde de har aftalt.\\
% Det har konsekvens for deres KPI'er, på denne måde\\

% In order to be able to meet a deadline, some prerequisits must be present...\\
% As stated above, the workload has recently been more formally delegated out between different teams. Therefore a lot of processes are being put in place and reinforced, in order to support this change. The general way of communication and sharing information inside MiL is done on Podio, in different workspaces assigned to each group. (!!PODIO Analysis? - say-do senarios?)

% With the new structure, in the future work practises should look like this. Work is now mostly done within the individual teams. The newly appointed team leaders will then meet every one and a half month to discuss what decisions have been made and talk about upcoming important events and deadlines. All other communication, and needed information should be available and shared on Podio, such that individual teams can act independently.\\

% There is a clear ambition that Podio should used extensively, and a feeling that it fits the needs of MiL. At this moment though, different teams use other technologies, Google Drive etc, on the side of Podio(kilde?). This spreads out information, making it unobtainable for some members. \

% The board also included a short presentation of Podio at the General Assembly. As stated in appendix xx the board wants to have all internal communication in the planning group at Podio. This was clearly a new tool for some of the leaders, which was stated in the form of different questions(kilde?). Most of the presentation was about the different subpages at Podio and how to create things like events and tasks. This gave some insight what Podio is cable of, but no introduction was given about when a task was done nor when a task should be created for another member (better examples?).\\

% (Har vi noget om hvad der før har været af regler for beslutninger? 5000kr og over skal forbi bestyrelsen? For at beskrive behov for at synligøre større beslutninger?)\\

% During the interview with Stine F the project group made her participate in a 'process'-game. Here the interviewers asked her to explain the task of booking volunteers for the festival. The result of this game can be seen in appendix xx. In our opinion it clearly shows that a lot can be done in termns of deadlines for bookings of volunteers and also the way its done.

Throughout the planning process, work is done at odd hours, as it is done along side peoples jobs and education. There is no MiL office, work is done at where ever suits the arrangers, and the setting of meetings is decided from time to time, making it formal and informal as needed. Generally, work can be customized to fit the needs of the arranger doing it, .
\\ \\
Teams work independently with designated areas of responsibility. They communicate by using Podio, but also extensively by means of informal communication such as telephone, face to face meetings and email. Arrangers learn how to use Podio by trial and error, as no training material exists. Communication between different teams as well as within a single team is documented in a an ad-hoc manner, and no formal requirements or strategy of documentation exists. Communication about a subject can be found on different posts in the activity stream, as comments and responses to questions. 
\\
Decisions made outside of schedule meetings are also documented in a very ad-hoc manner, and no formal requirements or strategy of documentation exists. If an urgent matter arises, that can not wait until a meeting, this can be communicated as a status in the stream of the team, and a decision of what needs to be done can be taken in the comments, and an assignee to realize the work can be chosen. This may be turned into a task, via the Podio app of the same name, and documented this way.
\\ \\
For formal meetings, such as the scheduled team leader meetings, a Podio app for creating meeting minutes exist. The app provides means of recording agenda, planned meeting date, attendees and minutes of the meeting. In addition, its possible to add tags which makes searching of minutes and agendas easier. Using these is not a requirement. Deadlines agreed on at these meetings are recorded in the minutes, this can also spawn tasks from the task app. No guidelines for defining when a task is completed exist, nor possible in the app, that is, definitions of when a a task is completed is made by the arranger to whom the task is assigned, often implicitly. When a task is completed, the responsible arranger reports this to other arrangers who she deems needs knowledge of the task completion. If an arranger is in doubt of the date of a deadline, she will search for the information on Podio.
\\ \\
Communication on Podio is carried out informally. No arranger has the role of moderator, and discussions are allowed to go off-topic without any restrictions. Some post are social, and does not have to do with MiL. Arrangers have agreed on a 24-hour Podio rule, which dictates that questions posted on Podio must receive a response within 24 hours.

\textbf{Stolen from above!:} As mentioned earlier Podio is used as the primary communication platform, for all levels of internal communication in the organization. Because of this, it is important that every member of the organization knows how to use Podio, but also frequently checks for updates and news, to being able to track the progress of the work. It is therefore a requirement that all members read posts directed directly at their area of work within 24 hours. 

\section{Goals, problems, and needs}
\label{sec:goprne}
In this section we will summarize the goals, problems, and needs that the
in-depth phase analysis has lead us to.

\subsection{Goals}
\label{subsec:goals}
\begin{itemize}
    \item It is clear to us, according to \ref{interview source} that it is a goal for \mil  to cut down the amount of missed deadlines.
    \item Some parts of the board \ref{Kristian source} sees it as a goal that the
    budget is not only kept, but that the offers collected on expensive things, are
    also the best offers.
    \item It is a goal for the board that Podio is the single point of communication
    for the arranging of \mil.
\end{itemize}

\subsection{Problems}
\label{subsec:problems}
According to our analysis \ref{source}, some problems are made clear.
\begin{itemize}
    %\item It is neither clearly stated when different tasks are due to internally in the different groups nor between the working groups.
    %\item People leaves out some information on Podio, because they do not know how to add it properly.
    \item There is no single place on Podio where all important deadlines are put.
    \item There is no clear guidelines for when a task is done.
	 \item Knowledge is lost when arrangers leaves the organization
	 \item Decisions is not documented on Podio
	 \item No clear guidelines for what a task at Podio involves
	 \item Information is lost in the feed at Podio, due to posts with non-relevant information
	 \item Arrangers not aware of deadlines with other groups
    \item (Add more if any...)
	 
\end{itemize}

\subsection{Needs}
\label{subsec:needs}
After several interviews with different people from \mil, it is clear to us that
there are some common needs.
\begin{itemize}
    \item According to \ref{source} there is a need for a system that makes it
    clear to everyone which deadlines are due to when.
    \item Guidelines describing when a task is done are also needed
    \ref{source}.
    \item A common understanding and knowledge of how the basics of how Podio
    works, assuring that everyone is able to document the progress of their work
    and tasks is also needed \ref{source / General Assembly Appendix}.
\end{itemize}

\section{Ideas for solutions}
\label{sec:ideas}
In this section we will sum up the ideas that we have come up with for solutions
that can help solve some of the problems we discovered in section \ref{sec:goprne}.

\subsection{Podio workshop}
It has come to our attention in section \ref{subsec:needs} that
some of the new volunteers have little knowledge of how Podio works. Due to
this, one of our suggestions is to make a Podio workshop that teaches the basics
of Podio the way the board of \mil has in mind that is should be used.

\subsection{Podio tutorial}
In addition to this, a tutorial made in Podio that uses the features of Podio to
teach how Podio works, is also a solution that we believe to be very useful.

\subsection{Podio App}
Since it is a goal in section \ref{subsec:goals} that all communication should take place
on Podio, we would like to make a Podio app that allows the user to make tasks
and dependencies for each area of responsibility in the different groups.

\subsection{Done and done done}
There is an internal goal in the board described in section \ref{subsec:goals} regarding the
budget that we think can be solved by making some clear guidelines describing
the procedure for the gathering of offers exceeding a price of XXXX DKK. A
detailed description of the solution can be found in appendix \ref{solution
appendix}

\subsection{Wiki (THIS PART SHOULD MAYBE BE DELETED?)}
We have also found out that it is a pain \ref{source to interview} to have a
common place to share knowledge between the groups. A possible solution to this
is to make an app that makes it really easy to add knowledge to a shared wiki
through Podio.

