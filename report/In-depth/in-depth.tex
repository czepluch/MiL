<<<<<<< Local Changes
=======
%!TEX root = /Users/Abj/git/MiL/report/report.tex
\part{In-depth Analysis}
\label{prt:in_depth_analysis}
\section{Organizational setup, key players and KPI's}
The key players, concerned with this project, are the ones that have the highest level of interest, and the abilities to perform the needed changes towards its goal, such that the different levels of KPI's can be met. The different sections of the organisations described in the work domains \ref{sec:work_domains}, have different impacts on the realisation of deadlines, as expressed by the mapping in \ref{sec:stakeholder_map}.
\\ \\
The board has great interest in the status of task and their deadlines, as this is a tool for them to get an overview of the progress of the planning of the festival. This overview is essential to the board, as it enables them to track that the strategies are being kept, and that the different teams are keeping their budgets. This insures that the overall budget is kept, making ground for the financial stability that is a clear goal of the board \ref{i2q1} \ref{i1q3}, so that they can keep their current value propositions \ref{sub:value_propositions}.
\\ \\
With the importance of the budgets, the team leaders gets a lot of power, as they are generally the ones responsible for their own teams budgets, and they are the ones that have the power to keep it \ref{i3q1}. This comes down to, the team leaders being the primary owners of the different deadlines, both internally in the team, but also the ones agreed upon between teams, also making them one of the parties most interested in keeping the deadlines.
\\ \\
The other arrangers in the different teams, do not have as much interest in keeping the deadlines, as this is not their responsibility. They do however have nearly as much power to realise the deadlines by completing tasks, but lack the cross-team influence. \ref{i4q9}


\section{Current work practices}
Throughout the planning process, work is done at odd hours, as it is done along side peoples jobs and education. There is no MiL office, work is done at where ever suits the arrangers, and the setting of meetings is decided from time to time, making it formal and informal as needed. Generally, work can be customized to suit the needs of the arranger doing it\ref{podio_struc}.
\\ \\
Teams work independently with designated areas of responsibility \ref{i2q9}. They communicate by using Podio \ref{i1q5}, but also extensively by means of informal communication such as telephone, face to face meetings and email \ref{i4q6} \ref{i2q11}. Arrangers learn how to use Podio by trial and error, as no training material exists. Communication between different teams as well as within a single team is documented in a an ad-hoc manner, and no formal requirements or strategy of documentation exists. Communication about a subject can be found on different posts in the activity stream, as comments and responses to questions\ref{podio_com}. 
\\
Decisions made outside of schedule meetings are also documented in a very ad-hoc manner, and no formal requirements or strategy of documentation exists. If an urgent matter arises, that can not wait until a meeting, this can be communicated as a status in the stream of the team, and a decision of what needs to be done can be taken in the comments, and an assignee to realize the work can be chosen\ref{podio_decis}. This may be turned into a task, via the Podio app of the same name, and documented this way \ref{podio_doc}.
\\ \\
For formal meetings, such as the scheduled team leader meetings, a Podio app for creating meeting minutes exist. The app provides means of recording agenda, planned meeting date, attendees and minutes of the meeting. In addition, its possible to add tags which makes searching of minutes and agendas easier. Using these is not a requirement. Deadlines agreed on at these meetings are recorded in the minutes, this can also spawn tasks from the task app. No guidelines for defining when a task is completed exist, nor possible in the app, that is, definitions of when a task is completed is made by the arranger to whom the task is assigned, often implicitly. When a task is completed, the responsible arranger reports this to other arrangers who she deems needs knowledge of the task completion. If an arranger is in doubt of the date of a deadline, she will search for the information on Podio\ref{i4q5}\ref{i4q6}.
\\ \\
Communication on Podio is carried out informally. No arranger has the role of moderator, and discussions are allowed to go off-topic without any restrictions \ref{i2q11}. Some post are social, and does not have to do with MiL\ref{podio_subj}. Arrangers have agreed on a 24-hour Podio rule, which dictates that questions posted on Podio must receive a response within 24 hours.\ref{ws1}

\section{Goals, problems, and needs}
\label{sec:goprne}
In this section we will summarize the goals, problems, and needs that the
in-depth phase analysis has lead us to.

\subsection{Goals}
\label{subsec:goals}
\begin{itemize}
    \item Tasks for which dependencies exist across arranger teams, should be well defined both in terms what needs to be done, and when it should be done. In addition, these should be visible on Podio.
    \item Work should be documented on Podio by the arrangers, so that arrangers and team leaders, who's work is dependent on the work of others, can anticipate delays and take appropriate action.
    \item Questions and requests made on Podio should be acknowledges within 24 hours.
    \item Off-topic discussions should be separated from other communication, such that finding relevant information can be done efficiently.
    \item Podio should be used as the primary means of communication, so that arrangers and team leaders can find information quickly and easily.
\end{itemize}

\subsection{Problems}
\label{subsec:problems}
\begin{itemize}
    %\item It is neither clearly stated when different tasks are due to internally in the different groups nor between the working groups.
    %\item People leaves out some information on Podio, because they do not know how to add it properly.
    \item There is no single place on Podio where all important deadlines are put.
    \item There is no clear guidelines for when a task is done.
	 \item Knowledge is lost when arrangers leaves the organization
	 \item Decisions is not documented on Podio
	 \item No clear guidelines for what a task at Podio involves
	 \item Information is lost in the feed at Podio, due to posts with non-relevant information
	 \item Arrangers not aware of deadlines with other groups, and the dependencies between them
	 
\end{itemize}

\subsection{Needs}
\label{subsec:needs}
\begin{itemize}
    \item Arrangers and team leaders need guidelines on how to use Podio, such that information on the platform is well structured and accessible.
    \item Arrangers and team leaders need tools for quickly and easily documenting their work, and reporting progress on tasks that other teams depend on.
    \item Arrangers and team leaders need infrastructure for defining tasks for other teams. (skal lige formuleres ordentligt).
\end{itemize}

\section{Ideas for solutions}
\label{sec:ideas}
In this section we will sum up the ideas that we have come up with for solutions
that can help solve some of the problems we discovered in section \ref{sec:goprne}.

\subsection{Podio guidelines}
A solution that goes along with the goal of continuing to use Podio, without changing the work practices of the arrangers, would be to optimize their use of Podio. This would be done be educating the arrangers better about the platform, and setting some guidelines for the communication. Better use of Podio could secure clearer communication, reduce amount of information loss due to lack of structure.

\subsection{Podio App}
Podio have much of the functionality needed to solved the problems uncovered in this section, and allows for further extension. A solution could be adding an app that extends the functionality of the Task app, to better support dependencies between tasks and visualising these, along with making it easier to define the extend of a task and the requirements for calling it done. This would still keep the goal of keeping Podio as the main platform, and extend on a technology the arrangers already know.

\subsection{Other Another platform for communication and sharing}
The problem of information getting lost in the activity stream on Podio, can only to some extend be solved. This is quite an integrated part of how Podio works. The only solution that completely solves this, would be to move to a different platform. This would though go against the wishes and goal of Musik i Lejet, and to completely eliminate the loss of information in communication, would mean to cut down on the informal communication that is a big part of their work practises.

>>>>>>> External Changes
