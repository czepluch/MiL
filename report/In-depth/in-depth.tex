%!TEX root = /Users/Abj/git/MiL/report/report.tex
\part{In-depth Analysis}

\section{Organizational setup, key players and KPI's}
\subsection{Organizational setup}
\subsubsection{Structural Change}
Before the project period was started the organization initiated a structural change. This influenced the board, but mostly the planning group, since this was expanded from X to Y members. This change was conducted in order to move and spread the knowledge and responsibility from the board across the organization, by delegating responsibility to leaders in different planning teams(interview kilde).

\subsection{Key Players}
Several types of key players can be identified in the organisation. Before the structural change, the level of key player, was determined by each persons experience with planning of Mil. Naturally, most of the key players was also part of the board.\\
With the newly structural change, these players are made responsible for a team, which enforces two things. Firstly, a lot of the planning are moved away from the board. Second, when delegating a person as a leader for a team, the knowledge for this particular area of planning is more assigned to a role in a team instead of a specific person. The consequences of this change, is that more key players can be identified, which creates more dependencies between the different area of planning, since more people are involved(kilde på at flere mennesker skaber mere afhængigheder).\\
Thereby, the current most interest key players are the leaders of each teams, which the stakeholder analysis in appendix xx suppports.\\
Bilag: STAKEHOLDER ANALYSIS\\

\subsection{Key Performance Indicators}
Top level KPI -> deadlines eksempel\\

Bilag:\\
Top level KPI's\\
finanical Stability\\
	- Ticket price\\
	- keeping budget\\
Podio\\
- 24 hour rule\\

KPI's: Ticket price, financial stability, keeping the budgets, 24 hour rule of podio\\

At this point of the project we identified a number of KPI's and we will use this section to underline some of the aspects according to these.\\
As described in appendix xx MiL has several top level KPI's which is crucial in respect to customer satisfaction and the overall sale during the festival(kilde?). 

\section{Current work practices}
As stated above, the workload has recently been more formally delegated out between different teams. Therefore a lot of processes are being put in place and reinforced, in order to support this change. The general way of communication and sharing information inside MiL is done on Podio, in different workspaces assigned to each group. (!!PODIO Analysis? - say-do senarios?) \

With the new structure, in the future work practises should look like this. Work is now mostly done within the individual teams. The newly appointed team leaders will then meet every one and a half month to discuss what decisions have been made and talk about upcoming important events and deadlines. All other communication, and needed information should be available and shared on Podio, such that individual teams can act independently.\\

There is a clear ambition that Podio should used extensively, and a feeling that it fits the needs of MiL. At this moment though, different teams use other technologies, Google Drive etc, on the side of Podio(kilde?). This spreads out information, making it unobtainable for some members. \

The board also included a short presentation of Podio at the General Assembly. As stated in appendix xx the board wants to have all internal communication in the planning group at Podio. This was clearly a new tool for some of the leaders, which was stated in the form of different questions(kilde?). Most of the presentation was about the different subpages at Podio and how to create things like events and tasks. This gave some insight what Podio is cable of, but no introduction was given about when a task was done nor when a task should be created for another member (better examples?).\\

(Har vi noget om hvad der før har været af regler for beslutninger? 5000kr og over skal forbi bestyrelsen? For at beskrive behov for at synligøre større beslutninger?)\\

During the interview with Stine F the project group made her participate in a 'process'-game. Here the interviewers asked her to explain the task of booking volunteers for the festival. The result of this game can be seen in appendix xx. In our opinion it clearly shows that a lot can be done in termns of deadlines for bookings of volunteers and also the way its done.

\section{Goals, problems, and needs}
\label{sec:goprne}
In this section we will summarize the goals, problems, and needs that the
in-depth phase analysis has lead us to.

\subsection{Goals}
\label{subsec:goals}
\begin{itemize}
    \item It is clear to us, according to \ref{interview source} that it is a goal for \mil  to cut down
    the amount of missed deadlines.
    \item Some parts of the board \ref{Kristian source} sees it as a goal that the
    budget is not only kept, but that the offers collected on expensive things, are
    also the best offers.
    \item It is a goal for the board that Podio is the single point of communication
    for the arranging of \mil.
\end{itemize}

\subsection{Problems}
\label{subsec:problems}
According to our analysis \ref{source}, some problems are made clear.
\begin{itemize}
    \item It is neither clearly stated when different tasks are due to
    internally in the different groups nor between the working groups.
    \item People leaves out some information on Podio, because they do not know
    how to add it properly.
    \item There is no single place on Podio where all important deadlines are
    put.
    \item There is no clear guidelines for when a task is done.
    \item (Add more if any...)
\end{itemize}

\subsection{Needs}
\label{subsec:needs}
After several interviews with different people from \mil, it is clear to us that
there are some common needs.
\begin{itemize}
    \item According to \ref{source} there is a need for a system that makes it
    clear to everyone which deadlines are due to when.
    \item Guidelines describing when a task is done are also needed
    \ref{source}.
    \item A common understanding and knowledge of how the basics of how Podio
    works, assuring that everyone is able to document the progress of their work
    and tasks is also needed \ref{source / General Assembly Appendix}.
\end{itemize}

\section{Ideas for solutions}
In this section we will sum up the ideas that we have come up with for solutions
that can help solve some of the problems we discovered in section \ref{sec:goprne}.

\subsection{Podio workshop}
It has come to our attention in section \ref{subsec:needs} that
some of the new volunteers have little knowledge of how Podio works. Due to
this, one of our suggestions is to make a Podio workshop that teaches the basics
of Podio the way the board of \mil has in mind that is should be used.

\subsection{Podio tutorial}
In addition to this, a tutorial made in Podio that uses the features of Podio to
teach how Podio works, is also a solution that we believe to be very useful.

\subsection{Podio App}
Since it is a goal in section \ref{subsec:goals} that all communication should take place
on Podio, we would like to make a Podio app that allows the user to make tasks
and dependencies for each area of responsibility in the different groups.

\subsection{Done and done done}
There is an internal goal in the board described in section \ref{subsec:goals} regarding the
budget that we think can be solved by making some clear guidelines describing
the procedure for the gathering of offers exceeding a price of XXXX DKK. A
detailed description of the solution can be found in appendix \ref{solution
appendix}

\subsection{Wiki (THIS PART SHOULD MAYBE BE DELETED?)}
We have also found out that it is a pain \ref{source to interview} to have a
common place to share knowledge between the groups. A possible solution to this
is to make an app that makes it really easy to add knowledge to a shared wiki
through Podio.

