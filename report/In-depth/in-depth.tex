%!TEX root = /Users/Abj/git/MiL/report/report.tex
\part{In-depth Analysis}

\section{Organizational setup, key players and KPI's}
At this point of the project we identified a number of KPI's, during our observations and interviews and we will use this section to underline some of the aspects according to these.
In appendix xx a summary of an interview with Stine F is being described and she talks about some of her concerns about the price of a ticket to the festival. She believes that \\
KPI:\\
-Billetpris/økonomisk stabilitet\\
-Teams skal overholde deres budget
- 24 timers regel på podio
\section{Current work practices}
This section will describe some of the work there is, when working in the planning group of Mil. Before going in to the actual descriptions, we do however need to mention, that observing work practices in an organization as this, is not a trivial task. We have observed the work practices in the form of interviews and an observation at the General Assembly, which hardly can be formulated as actual practices, but more as a mean to work. However, we have found several things, which we believe can be used as documentation for the practices. \\
At the General Assembly a suggestion from the board was raised, about all members being charged a fee for being part of MiL. The board explained that this was more a matter of regulation of a union, and to get some financial advantages, than a need for payment. The non-board(?) members raised some concerns regarding this issue, as stated in appendix xx. Surprisingly, even though the non-board members raised strongly concerns, the board decided to proceed with the suggestion, without a voting about it. \\
The board also included a short presentation of Podio at the General Assembly. As stated in appendix xx the board wants to have all internal communication in the planning group at Podio. This was clearly a new tool for some of the leaders, which was stated in the form of different questions(kilde?). Most of the presentation was about the different subpages at Podio and how to create things like events and tasks. This gave some insight what Podio is cable of, but no introduction was given about when a task was done nor when a task should be created for another member (better examples?).\\

During the interview with Stine F the project group made her participate in a 'process'-game. Here the interviewers asked her to explain the task of booking volunteers for the festival. The result of this game can be seen in appendix xx. In our opinion it clearly shows that a lot can be done in termns of deadlines for bookings of volunteers and also the way its done.

- Stemning og struktur på General Forsamling (blandt andet forslag om kontigent blev ikke besluttet)\\
- Podio (manglende af done og done-done) og noget andet?\\
- Problem med høj overpris af hegn\\
- Beskrivelse af process spil

\subsection{Goals, problems, and needs}
In this section we will summarize the goals, problems, and needs that the
in-depth phase analysis has lead us to.

\subsubsection{Goals}
It is clear to us, according to \ref{interview source} that it is a goal for \mil to cut down
the amount of missed deadlines.
\\
Some parts of the board \ref{Kristian source} sees it as a goal that the
budget is not only kept, but that the offers collected on expensive things, are
also the best offers.
\\
There is also a certain desire \ref{Learn podio reference} that everyone knows
how to use Podio, since the board demands that Podio is the single point of
communication for \mil.

\subsubsection{Problems}
According to our analysis \ref{source}, some problems are made clear. \\
              \begin{itemize}
    \item It is neither clearly stated when different tasks are due to
    internally in the different groups nor between the working groups.
    \item People leaves out some information on Podio, because they do not know
    how to add it properly.
    \item There is no single place on Podio where all important deadlines are
    put.
    \item There is no clear guidelines for when a task is done.
    \item (Add more if any...)
\end{itemize}

\subsubsection{Needs}
After several interviews with different people from \mil, it is clear to us that
there are some common needs.
\begin{itemize}
    \item According to \ref{source} there is a need for a system that makes it
    clear to everyone which deadlines are due to when.
    \item Guidelines describing when a task is done are also needed
    \ref{source}.
    \item A common understanding and knowledge of how the basics of how Podio
    works, assuring that everyone is able to document the progress of their work
    and tasks is also needed \ref{source}.
\end{itemize}


\subsection{Ideas for solutions}
In this section we will discuss some of the ideas we have for technical
solutions to help ease the organisational restructuring of \mil.

\subsubsection{Podio workshop}
It has come to our attention when attending \ref{General Assembly Appendix} that
a lot of the volunteers have very little knowledge of how Podio works. Due to
this, one of our suggestions is to make a Podio workshop that teaches the basics
of Podio the way the board of \mil has in mind that is should be used.

\subsubsection{Podio tutorial}
In addition to this, a tutorial made in Podio that uses the features of Podio to
teach how Podio works, is also a solution that we believe to be very useful
\ref{reference to something about how good learning hands on is} in the process
of learning how to use Podio.

\subsubsection{Podio App}
Since it is a goal \ref{source to this} that all communication should take place
on Podio, we would like to make a Podio app that allows the user to make tasks
and dependencies for each area of responsibility in the different groups.

\subsubsection{Done and done done}
There is a concern amongst some of the board members \ref{source from meeting
with Andreas} that since each responsibility group now have their own budget to
administer they will not necessarily use energy to find the best offers on the
market. Our solution proposition is to make a rule that when a person gets the
responsibility to find offers on eg the fence surrounding the festival area, the
person has to collect three offers and the rest of the responsibility group now
has to decide which one is best. The task to find an offer on the fence is now
done, but it is not marked as done done before the contract is signed.

\subsubsection{Wiki}
We have also found out that it is a pain \ref{source to interview} to have a
common place to share knowledge between the groups. A possible solution to this
is to make an app that makes it really easy to add knowledge to a shared wiki
through Podio.

