\part{In-line Analysis}

\section{Business Strategy}
The main objective of Musik i Lejet is to provide a cultural offer in the local community of Tisvildeleje, both as an offer to the local populous, and also to attract business to the community. This is achieved by a non-profit, cost driven business model, that relies heavily on volunteer work, grant money and private sponsorship. Cultural and economic value is generated by offering musical entertainment for a variety of customer segments, such as music enthusiasts, families with children and customers interested in attending parties.
\\ \\
Keeping a well defined and contemporary musical profile is essential for Musik i Lejet to fulfill both its cultural objectives, and to make the festival competitive with other festivals of the same size, addressing the same customer segments. Moreover, diversity within the festivals audience is of huge importance to Musik i Lejet. This is in part achieved by offering a wide selection of music in many different genres, but also by keeping ticket prices low, and providing free admission for children and the elderly.
\\ \\
The main challenges for Musik i Lejet include maintaining relationships with and interest from sponsors and cultural foundations, securing the necessary number of volunteer workers, coordinating the efforts of the 20-30 people who, without a physical and centralized work space, arrange the festival in their spare time, and obtaining contracts with contemporary and popular performing artists to an affordable price.

\section{Environment}
When studying MiL it is of importance to view it in the relative perspective of its surroundings to be able to determine strengths/weaknesses as well as relations, ultimately to support potential solutions. In this case we deal with the remaining forces of danish music festivals.
\\  \\
The industry of danish music festivals is a healthy and growing market showing indications of trends towards an increasing appreciation of larger live events. Many festivals exists as of today, but differentiate significantly in vision, identity, and size. This means that when considering the market, competition to MiL can not be inferred if they do not target the same customer segments. Due to MiLs price and target segments, opposing threats are few in numbers, e.g. Wonderfestiwall and trailerpark festival. This compitition fails to compete with MiL by due to the significent difference in ticket price.
\\  \\
Stable relationships with providers and stakeholders are of big part of what makes MiL possible, and are therefore an important aspect to maintain and care. Due to success throughout MiLs past years of events can provide stability as an organisation through factors such as: sold out events, positive publicity and continous increase in revenue. With this MiL has established quality cooperation agreements providing security and avoiding risk at depencies.
\\  \\
A potential risk occuring as a product of MiLs increasing success is the organisations capability of scaling its internal resources to match a larger event while maintaining the same value offer - This in view of the environment MiL lacks experience due to its relative young age. Failing to being able to cope with expansion could result in increased ticket prices. This would remove MiLs advantage, enabling competition to become competition


\section{Work Domains}
\label{sec:work_domains}
\subsection{The board}
\label{sub:the_board}
The main responsibility of the board is to make all major decisions regarding the execution of the
festival. They are also responsible for all strategic decisions and they handle the booking of
some of the big artists and the approval of expensive contracts and deals that the teams make.

\subsection{The team leaders}
\label{sub:team_leaders}
The team leaders each have an area of responsibility which results in a team leader being
responsible and the leader of a team of volunteers that focus on a specific area of the festival
planning. As seen in the \ref{sub:organisation} there are 12 independent teams. The most important
role of a team leader is to communicate with the other team leaders and give feed back to the board
regarding the status of the team's current tasks. Team leaders meeting are held once every ?????,
where the team leaders give status updates to each other and discuss inter team problems and
deadlines.

\subsection{The team members}
\label{sub:team_members}
A team member is simply a volunteer that help out with the planning of the festival, and not only
the execution. As a member of a team you will be given different tasks accomplish depending on which
team you are in. If you are in the bar team, a task could be to make sure that offers for bar tents
have been obtained before a certain deadline. 

