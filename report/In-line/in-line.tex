\part{In-line Analysis}


\section{Business Strategy}
The main objective of Musik i Lejet is to provide a cultural offer in the local community of Tisvildeleje, both as an offer to the local populous, and also to attract business to the community. This is achieved by a non-profit, cost driven business model, that relies heavily on volunteer work, grant money and private sponsorship. Cultural and economic value is generated by offering musical entertainment for a variety of customer segments, such as music enthusiasts, families with children and customers interested in attending wild parties.
\\ \\
Keeping a well defined and contemporary musical profile is essential for Musik i Lejet to fulfill both its cultural objectives, and to make the festival competitive with other festivals of the same size, addressing the same customer segments. Moreover, diversity within the festivals audience is of huge importance to Musik i Lejet. This is in part achieved by offering a wide selection of music in many different genres, but also by keeping ticket prices low, and providing free admission for children and the elderly.
\\ \\
The main challenges for Musik i Lejet include maintaining relationships with and interest from sponsors and cultural foundations, securing the necessary number of volunteer workers, coordinating the efforts of the 20-30 people who, without a physical and centralized work space, arrange the festival in their spare time, and obtaining contracts with contemporary and popular performing artists to an affordable price. In addition to this, Musik i Lejet must manage its relationship with neighbors of the festival site, as these apply political pressure for terminating the activities of the organization.

\section{Environment}
The industry of danish music festivals is a growing market attracting more people each year and has shown indications of a trend towards increased appreciation of larger live events. These different festivals differentiate significantly in vision, identity, and as mentioned, size. This means that when considering the market, competition does not exist on a higher level between all organisations of the market, but rather “locally” between festivals that seek to target the same customer segments. With this in notion music festivals also have a significant historical aspect as of which events range over a great deal of lifespans, some beyond decades, some newborn. With experience as a product to age, the ability to adapt becomes a value for the individual events. 

\bigskip

With a ticket price of 400 kr. for 3 full days of music and the distinct music profile, direct competing festival alternatives are significantly few in numbers, suggesting events such as trailerpark festival, or spot festival. MiL, being a relatively young music event, having an impressive amount success, might then not seem to points towards threats in external factors, such as competition or market issues, but rather internal challenges in relation to the increasing size of the organisation (customers, budget, and expectations). 

\bigskip

/*
The range of festivals of the market spans over a wide range of types of events, from heavy rock festivals like Copenhell(link) with a guest count of 11.000, to the multigenre Roskilde festival with a size of 100.000 guests(link). 
*/


\section{Work Domains}

In the summer of 2009 the MiL festival was hosted the first time, organized and executed by a small handful of people. When MiL was approached in the fall of 2013 it was made clear that they were going through a restructuring phase of the organisation.

\smallskip

Prior to the restructuring of the MiL the organisation had a fluent and dynamic type of structure. The board consisted of 7 members, accompanied by an organizer-group of X people, in total representing the backbone and of the festival. This delegation functioned as a 1-tier hierarchy whereof each person of the backbone is in charge of a single and/or multiple areas of responsibility, this spans unfiltered between board members and members of the organizer-group. This applied to pre-event workloads as well as on-sight festival execution tasks.

\smallskip

A plan of restructuring MiL was introduced in the spring of 2013. This reduced the amount of board members from 7 to 4, now introducing 12 independent teams with fixed members and individual team leaders, thus introducing distinct roles and modulated responsibility. From the previous single leveled structure, the organisation would now be represented in a classical tree-structure with several hierarchical levels, instead of a single one - with the board at the top with MiL high order decisions, and lower tiers with specified detailed decisions.

