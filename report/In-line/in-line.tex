\part{In-line Analysis}

\section{Business Strategy}
The main objective of Musik i Lejet is to provide a cultural offer in the local community of Tisvildeleje, both as an offer to the local populous, and also to attract business to the community. This is achieved by a non-profit, cost driven business model, that relies heavily on volunteer work, grant money and private sponsorship. Cultural and economic value is generated by offering musical entertainment for a variety of customer segments, such as music enthusiasts, families with children and customers interested in attending wild parties.
\\ \\
Keeping a well defined and contemporary musical profile is essential for Musik i Lejet to fulfill both its cultural objectives, and to make the festival competitive with other festivals of the same size, addressing the same customer segments. Moreover, diversity within the festivals audience is of huge importance to Musik i Lejet. This is in part achieved by offering a wide selection of music in many different genres, but also by keeping ticket prices low, and providing free admission for children and the elderly.
\\ \\
The main challenges for Musik i Lejet include maintaining relationships with and interest from sponsors and cultural foundations, securing the necessary number of volunteer workers, coordinating the efforts of the 20-30 people who, without a physical and centralized work space, arrange the festival in their spare time, and obtaining contracts with contemporary and popular performing artists to an affordable price. In addition to this, Musik i Lejet must manage its relationship with neighbors of the festival site, as these apply political pressure for terminating the activities of the organization.

\section{Environment}
The industry of danish music festivals is a growing market attracting more people each year and has shown indications of a trend towards increased appreciation of larger live events. These different festivals differentiate significantly in vision, identity, and as mentioned, size. This means that when considering the market, competition does not exist on a higher level between all organisations of the market, but rather “locally” between festivals that seek to target the same customer segments. With this in notion music festivals also have a significant historical aspect as of which events range over a great deal of lifespans, some beyond decades, some newborn. With experience as a product to age, the ability to adapt becomes a value for the individual events. 

\bigskip

With a ticket price of 400 kr. for 3 full days of music and the distinct music profile, direct competing festival alternatives are significantly few in numbers, suggesting events such as trailerpark festival, or spot festival. MiL, being a relatively young music event, having an impressive amount success, might then not seem to points towards threats in external factors, such as competition or market issues, but rather internal challenges in relation to the increasing size of the organisation (customers, budget, and expectations). 

\bigskip

/*
The range of festivals of the market spans over a wide range of types of events, from heavy rock festivals like Copenhell(link) with a guest count of 11.000, to the multigenre Roskilde festival with a size of 100.000 guests(link). 
*/


\section{Work Domains}
\label{sec:work_domains}
As of the autumn of 2013, the restructuring of \mil has taken place. This restructuring means
that there now is three layers in the organisational structure as seen in section
\ref{sub:organisation}:
\begin{itemize}
    \item The board
    \item The team leaders
    \item The team members
\end{itemize}
\subsection{The board}
\label{sub:the_board}
The main responsibility of the board is to make all major decisions regarding the execution of the
festival. They are also responsible for all the structural decisions and they handle the booking of
some of the big artists and the approval of expensive contracts and deals that the work groups make.

\subsection{The group leaders}
\label{sub:group_leaders}
The team leaders each have an area of responsibility which results in a team leader being
responsible and the leader of a team of volunteers that focus on a specific area of the festival
planning. As seen in the \ref{sub:organisation} there are 12 independent teams. The most important
role of a team leader is to communicate with the other team leaders and give feed back to the board
regarding the status of the team's current tasks. Team leaders meeting are held once every ?????,
where the team leaders give status updates to each other and discuss inter team problems and
deadlines.

\subsection{The team members}
\label{sub:team_members}
A team member is simply a volunteer that help out with the planning of the festival, and not only
the execution. As a member of a team you will be given different tasks accomplish depending on which
team you are in. If you are in the bar team, a task could be to make sure that offers for bar tents
have been obtained before a certain deadline. 

