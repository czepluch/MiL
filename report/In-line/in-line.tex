\part{In-line Analysis}

\section{Business Strategy}
The main objective of Musik i Lejet is to provide a cultural offer in the local community of Tisvildeleje, both as an offer to the local populous, and also to attract business to the community \ref{i1q3}. This is achieved by a non-profit, cost driven business model, that relies heavily on volunteer work, grant money and private sponsorship as seen in Appendix \ref{sec:business_model}. Cultural and economic value is generated by offering musical entertainment for a variety of customer segments, such as music enthusiasts, families with children and customers interested in attending parties \ref{sub:customer_segments}.
\\ \\
Keeping a well defined and contemporary musical profile is essential for Musik i Lejet to fulfill both its cultural objectives, and to make the festival competitive with other festivals of the same size, addressing the same customer segments. Moreover, diversity within the festivals audience is of huge importance to Musik i Lejet. This is in part achieved by offering a wide selection of music in many different genres, but also by keeping ticket prices low, and providing free admission for children and the elderly \ref{sub:value_propositions}.
\\ \\
The main challenges for Musik i Lejet include maintaining relationships with and interest from sponsors and cultural foundations, securing the necessary number of volunteer workers, coordinating the efforts of the 20-30 people who, without a physical and centralized work space, arrange the festival in their spare time, and obtaining contracts with contemporary and popular performing artists to an affordable price \ref{sub:key_resources} .

\section{Environment}
From a financial perspective, the customer segment taking advantage of the nightly after-parties is the most important, as the bar income at these events greatly exceeds the bar incomes during the concerts \ref{i2q5}. The customer segment labeled as music enthusiasts is also important to Musik i Lejet, as they purchase tickets in advance which contributes to economic stability \ref{i2q4}.
\\ \\
Musik i Lejets position in the markets mentioned above is threatened by competing festivals, particularly festivals with similar musical profile such as Trailerpark Festival. Musik i Lejet competes with these festivals by offering a unique atmosphere, created by the festival's location, and the visual identity created at the festival site \ref{i2q2} \ref{i2q4}.
\\ \\
Musik i Lejet's business model is highly dependent on the quality of the products and services delivered by suppliers and collaborators. If work delivered by musicians, service providers working at the festival, or stage technicians is of low quality, this will perceived as a loss of value by the customers directly. 

\section{Work Domains}
\label{sec:work_domains}
This section describes the large scale work domains of MiL, and how they relate to each other. The organizational structure found in appendix \ref{org_chart}, and which in part serves as the basis for this section, was put into effect in November of 2013. Before November 2013, Arranger teams as described in the following existed only informally.
\subsection{The board}
\label{sub:the_board}
The main responsibility of the board is to make strategic decisions of the festival. Arranger teams depend on these decisions for performing their tasks, as strategic decisions include e.g decisions on opening hours, beginning and end of music program. Thus, these strategic decisions can be described as constituting the framework in which the arranger teams work. The board is also responsible for creating the overall budget of the festival, which defines the limits of the budget for each of the arranger teams. A board member may also be part of one or more arranger teams as either an arranger or team leader. \ref{i2q8} \ref{i2q9} \ref{i4q4}

\subsection{Arranger teams} % (fold)
\label{sub:arranger_team}
An arranger team holds the responsibility for one coherent area of work in MiL as described in Appendix \ref{org_chart}. A team consist of a team leader, and possibly a number of arrangers. Some teams consist only of the team leader.
% subsection arranger_team (end)

\subsection{The team leaders}
\label{sub:team_leaders}
A team leader is the head of an arranger team. The team leader is responsible for communicating and coordinating with other teams who's work the team is dependent on or who is dependent on the work of the team lead by the team leader. Team leaders are also responsible for communicating with the board about the status of the team's work and whether or not the team is on budget. Team leaders from all arranger teams meet every 6 weeks, where leaders can update each other on the status of tasks, and coordinate tasks that are dependent on each other across teams. \ref{i3q1}

\subsection{Arrangers}
\label{sub:team_members}
An arranger works within an arranger team. Arrangers and team leaders in cooperation delegate responsibility of subtasks within the team. \ref{i4q9}

