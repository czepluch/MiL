\part{In-line Analysis}


\section{Business Strategy}
The main objective of Musik i Lejet is to provide a cultural offer in the local community of Tisvildeleje, both as an offer to the local populous, and also to attract business to the community. This is achieved by a non-profit, cost driven business model, that relies heavily on volunteer work, grant money and private sponsorship. Cultural and economic value is generated by offering musical entertainment for a variety of customer segments, such as music enthusiasts, families with children and customers interested in attending wild parties.
\\ \\
Keeping a well defined and contemporary musical profile is essential for Musik i Lejet to fulfill both its cultural objectives, and to make the festival competitive with other festivals of the same size, addressing the same customer segments. Moreover, diversity within the festivals audience is of huge importance to Musik i Lejet. This is in part achieved by offering a wide selection of music in many different genres, but also by keeping ticket prices low, and providing free admission for children and the elderly.
\\ \\
The main challenges for Musik i Lejet include maintaining relationships with and interest from sponsors and cultural foundations, securing the necessary number of volunteer workers, coordinating the efforts of the 20-30 people who, without a physical and centralized work space, arrange the festival in their spare time, and obtaining contracts with contemporary and popular performing artists to an affordable price. In addition to this, Musik i Lejet must manage its relationship with neighbors of the festival site, as these apply political pressure for terminating the activities of the organization.

\section{Environment}

\section{Work Domains}


