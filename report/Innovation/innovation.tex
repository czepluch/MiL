%!TEX root = /Users/Abj/git/MiL/report/report.tex
\part{Innovation}
As the previous sections of the \texttt{In-depth phase} in part \ref{prt:in_depth_analysis},
especially in section \ref{sub:goprne}, has shown,
the organisation has room for optimisation and improvements. Based on the \ref{sub:ideas} that we
have developed and the fact that the organisation wants to continue using Podio, we have chosen two that we believe will bring the most benefits to the organisation.
We will in the following analyse both the solutions, and try to determine which will be the best or
if a third solution will emerge from this analysis. A short description of the ideas in focus
follows:


1. Work processes for Podio\\
In the previous the Diagnostic Map in section xx revealed several problem and related causes found in the organization. In 5b its said that the cause of this problem, is that their is too much off topic discussion and also that no real difference between pure informational topics and topics for discussions are in place within the organizational use of Podio. As Stine F also points out in 1a, the organization faces a problem about sharing of knowledge, since this is not documented properly on Podio.\\

This solution proposal will attempt to structure work processes when using Podio and make guidelines for the information posted at Podio. This proposal will therefore be directed at the use of Podio, with respect to the mentioned things in the above, and will keep to the organizations strategy about using Podio as their primary IT solution. 

2. Podio App
This solution consists of the development of a Podio application \footnote{https://developers.podio.com/doc/app-store}.
The application will be used to create, store, and visualise tasks in- and externally in the teams.
Thus, hopefully, making it very transparent which tasks are due to when.\\

\section{WORK PROCESSES}
\subsection{Visions for change}
The visions for this solution proposal is a bit different, than what is normally suggested as a
solution, since this does not contain an actual implementation of a new IT solution, but rather a
change in the way the current system is used by the organisation.
Podio is very customisable and offers per default the possibility to create your own applications
suited for your needs. This solution will however focus solely on changing the way the organisation
is using their current available tools and work practices in Podio.


%The overall vision for the change, can be divided in to these subsections:
\subsubsection{Information at Podio}
As mentioned earlier, and seen on Podio, a lot information is non-relevant for the actual planning and should therefore be elsewhere. Based on the analysis, we believe that the reason for this non-relevant information is on Podio is due to the relaxed working environment shared by the members and also the lack of formal guidelines for what kind of information should be posted on Podio, by whom and when.\\ 
We suggest that the board makes a formal description of these, by categorizing them after importance.
This will impact every member of the planning group and board, since an equal amount of non-relevant information seems to be posted between all members. \\

Make scenario for how this new work practice will be for the persons impacted by it(p191)
  
\subsubsection{Sharing of knowledge}
With an organization structured and based highly upon volunteers like Musik I Lejet, there is always a risk of people leaving the organization with short notice and taking valuable information and knowledge with them. Since this risk is hard to limit, the solution proposal will try to embrace the knowledge that the members poses and expose this at relevant places on Podio, for others to use. The challenge with this, is to find the persons holding the knowledge and exploring it, since some of it may be tacit(use different levels of knowledge theory). A challenge is also to document this is a right way, so it can be reused at a later point and saving in a place on Podio where others will be able to find it.\\
The team leaders is probably the target for this part of the solution, since most of these a highly experienced and has a good feel about the different areas of the organization.\\

Make scenario for how this new work practice will be for the persons impacted by it(p191)

\subsubsection{Sharing of decisions and agreements}
Throughout the analysis phase, it has been pointed out that members of the organization find it hard to find out what decisions and agreements there have been made and where it is documented. Reasons for this is the causes mentioned in 2b and 4b in the Diagnostic Map in appendix xx. MORE STUFF HERE\\ 
Make scenario for how this new work practice will be for the persons impacted by it(p191)

\subsection{Technology}
Since this solution is based on the IT system that the organisation already uses, there is no point
in explaining the technology too much. A short summary of Podio can be found in the glossary in
section \ref{sec:glossary} and \ref{sec:technology}.

\subsection{Work organization}
As mentioned above, this solution, does not need an implementation of a new IT system. It does how
ever consist of changes to the work organisations of the organisation. 
We have concluded in section \ref{podio bloat source} that a lot of the information and
communication on Podio is irrelevant to the planning of the festival. It seems that a good solution
to this, would be to make some guidelines in cooperation with the board that states what kind of
communication should take place on Podio and where it should take place. If most of the irrelevant
comments and posts dissapeared, the change of overlooking important information and deadlines
decreases. 
* Frivillig organisation, de gør det også for at have det sjovt
* 

\subsection{Qualification needs}

\subsection{Advantages and disadvantages}

\subsection{Finances}

\subsubsection{Roll out}

\subsubsection{Training}
Forslag: Experts and normal users
\subsubsection{Data conversion}
Converting all the existing Podio data to be alligned with the new formal structure for information, knowledge and decisions...

\subsection{Implementation strategy}
The must effective way this solution can be implemented in the organization will be as a pilot project, where selected members of the organization will be trained in how to use the formal ... 



\section{PODIO APP}
\subsection{Visions for change}
As mentioned earlier, the organisation wants to continue to use Podio as their platform of
communication. This means that we will have to develop a solution based on their current IT system.
As far as we are concerned the changes will not be too significant, and since a lot of the users of
the system have very little experience with Podio, they will not experience a major change in their
work practices. 

Furthermore since the need of \mil does not require a big change in the work practices, but rather
just a structured way to manage certain tasks, it is very suitable to continue using Podio. (They
are sponsored by Podio and they cannot afford a customised IT system either.)

We will now run through our vision of the solution.

\subsection{Technology}
\label{sub:technology}
Since we, as mentioned in section \ref{sub:goprne} are bound to use Podio, the technology of choice
for our solution is naturally Podio. 

Podio allows the users to create and build their own applications, to use within their workspaces.
It does however require some skills and knowledge to do this, and if the application is required to
have some complicated functionality it might be preferred to get a professional to implement it.

This solution proposition builds on such an application for Podio that, in basic terms, makes it possible to add
tasks with a given deadline, dependency, and responsible person. A visualisation of the tasks with
other dependent tasks, must also be available within the application. 

\subsubsection{IT systems and IT platform}

As mentioned above the IT platform for this IT solution will be Podio. Podio is a good fit for an
organisation as \mil since it offers some great features regarding organisations, and it makes it
very easy to implement an organisational structure of workspaces. It also allows for the users to
build their own customised applications, which makes it possible for an organisation, to use a
standardised platform with a lot of basic functionality, to get their unique requirements fulfilled.
This is what makes Podio very suitable for an organisation like \milNO, where the budget is very
tight and it would require a lot of work to move all the already existing data to another system.

\subsubsection{Functions}
In the following a list of each functionality that the system will have is shown.
\begin{itemize}
    \item Creation of a task
    \item A task will have title, description, deadline, dependent task(s), owner, and responsible
    team.
    \item It must be possible to enable notifications to be received at a desired time before the
    deadline.
    \item Teams responsible for a dependent task, will be notified when another task is depending on
    it.
    \item It must be possible to see a visualisation of all tasks, in the form of a gantt chart.
    \item The board must be notified if a task does not satisfy a deadline.
    \item It must be possible to set a task as done.
    \item When a task is done, it must be possible to save the information documenting that the
    given task is done. (Ex a signed contract for the fence surrounding the festival area.)
\end{itemize}

\subsubsection{User interfaces}
Podio has its own user interface, so there are not many changes to make in this area. The members of
the organisation have been using Podio for some time, which makes them somewhat familiar with the
user interface of Podio. We see this as an advantage, since we do not need to spend too many
resources on making the users of Podio used to the Podio user interface.

\subsection{Work organisation}
\label{sub:work_organisation}
The adjustments to the work organisation is very limited in this case. There are however some
minor adjustments that are required for the solution to be implemented successfully. 
Most importantly is it, to properly educate all the members of the Podio organisation, in the use of
the application, thus making sure everyone knows that the application exists and how to use it. It
is especially important for the team leaders to know how the application works, since they will be
the ones who mainly creates tasks. 

\subsection{Qualification needs}
\label{sub:qualification_needs}
According to section \ref{sec:work_domains}, the team leaders are the people responsible of making
sure a team's task is done before the deadline is exceeded. The team leaders are also responsible
of an entire team of arrangers, which means that they meet with their team from time to time. This
makes the team leaders the most obvious and important user of all the functionalities of the new
system, thus making them the main users of the new system. 

The team leaders are the ones with the most experience in using Podio \ref{source}. Due to the above
the need of qualifications are most important to the team leaders. 

It is therefore important that the team leaders get the proper education regarding the new
application and its features. As described in section \ref{sub:technology}, the application is quite
simple and without too many complicated features. With this in mind, we believe that giving the team
leaders plus the four board members a full introduction, with an estimated duration of three hours,
and a manual to the application, should suffice to get the attendees to feel home in the use of the
application. We do also believe that the knowledge gained during this introduction, will make the
team leaders able to teach the rest of the team, how the application works.

\subsection{Advantages and disadvantages}
\label{sec:advantages_disadvantages}
In this section, we will provide an overview of the advantages and disadvantages of this particular
IT solution.


\begin{center}
    \begin{tabular}{ | p{7cm} | p{7cm} |}
    \hline
    \textbf{Advantages} & \textbf{Disadvantages}  \\ \hline
    Simple & Does not fulfill all needs  \\ \hline
    The organisation already knows Podio & There is a risk that people will not use it correct \\
    \hline
    Creates awareness of deadlines ex- and internally in the teams & Time consuming to learn \\ \hline
    Gives the board a good overview of the current status of the current tasks & none \\ \hline
    Offers a central place to store documentation of done tasks & It will take a long time to move
    all old documentation to the new system \\ \hline
    Add more... & Add more here too... \\ 
    \hline
    \end{tabular}
\end{center}


\subsection{Finances}

\subsection{Implementation strategy}


