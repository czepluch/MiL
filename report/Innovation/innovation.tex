%!TEX root = /Users/Abj/git/MiL/report/report.tex
\part{Innovation}
As the previous sections of the \texttt{In-depth phase} in part \ref{prt:in_depth_analysis},
especially in section \ref{sub:goprne}, has shown,
the organisation has room for optimisation and improvements. Based on the \ref{sub:ideas} that we
have developed and the fact that the organisation wants to continue using Podio, we have chosen two that we believe will bring the most benefits to the organisation.
We will in the following analyse both the solutions, and try to determine which will be the best or
if a third solution will emerge from this analysis. A short description of the ideas in focus
follows:

1. Work processes for Podio

2. Podio App
This solution consists of the development of a Podio application \hyperref[Podio application]{https://developers.podio.com/doc/app-store}. The
application will be used to create, store, and visualise task in- and external in the teams. Thus
hopefully making it very transparent which tasks are due to when.

\section{Visions for change}

\subsection{Technology}
\label{sub:technology}

\subsubsection{IT systems and IT platform}

\subsubsection{Functions}

\subsubsection{User interfaces}

\subsection{Work organisation}
\label{sub:work_organisation}

\subsection{Qualification needs}
\label{sub:qualification_needs}

\section{Advantages and disadvantages}

\section{Finances}

\section{Implementation strategy}


