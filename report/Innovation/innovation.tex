%!TEX root = /Users/Abj/git/MiL/report/report.tex
\part{Innovation}
As the previous sections of the \texttt{In-depth phase} in part \ref{prt:in_depth_analysis},
especially in section \ref{sub:goprne}, has shown,
the organisation has room for optimisation and improvements. Based on the \ref{sub:ideas} that we
have developed and the fact that the organisation wants to continue using Podio, we have chosen two that we believe will bring the most benefits to the organisation.
We will in the following analyse both the solutions, and try to determine which will be the best or
if a third solution will emerge from this analysis. A short description of the ideas in focus
follows:


1. Work processes for Podio\\
In the previous the Diagnostic Map in section revealed several problem and related causes found in the organization. In 5b its said that the cause of this problem, is that their is too much off topic discussion and also that no real difference between pure informational topics and topics for discussions are in place within the organizational use of Podio. As Stine F also points out in 1a, the organization faces a problem about sharing of knowledge, since this is not documented properly on Podio.\\

This solution proposal will attempt to structure work processes when using Podio and make guidelines for the information posted at Podio. This proposal will therefore be directed at the use of Podio, with respect to the mentioned things in the above, and will keep to the organizations strategy about using Podio as their primary IT solution. 

2. Podio App
This solution consists of the development of a Podio application \footnote{https://developers.podio.com/doc/app-store}.
The application will be used to create, store, and visualise tasks in- and externally in the teams.
Thus, hopefully, making it very transparent which tasks are due to when.\\

\texttt{WORK PROCESSES}
\section{Visions for change}
The visions for this solution proposal is a bit different, than what is normally suggested as a solution, since this does not contain a actual IT solution, but more a change in the use of the already existing system Podio.\\
The overall vision for the change, can be divided in to subsections:\\
\subsubsection{Information at Podio}
As mentioned earlier, and seen on Podio, a lot information is non-relevant for the actual planning and should therefore be elsewhere. Based on the analysis, we believe that the reason for this non-relevant information is on Podio is due to the relaxed working environment shared by the members and also the lack of formal guidelines for what kind of information should be posted on Podio, by whom and when.\\ 
We suggest that the board makes a formal description of these, by categorizing them after importance.
This will impact every member of the planning group and board, since an equal amount of non-relevant information seems to be posted between all members. \\
Make scenario for how this new work practice will be for the persons impacted by it(p191)
  
\subsubsection{Sharing of knowledge}
With an organization structured and based highly upon volunteers like \mil, there is always a risk of people leaving the organization with short notice and taking valuable information and knowledge with them. Since this risk is hard to limit, the solution proposal will try to embrace the knowledge that the members poses and expose this at relevant places on Podio, for others to use. The challenge with this, is to find the persons holding the knowledge and exploring it, since some of it may be tacit(use different levels of knowledge theory). A challenge is also to document this is a right way, so it can be reused at a later point and saving in a place on Podio where others will be able to find it.\\
The team leaders is probably the target for this part of the solution, since most of these a highly experienced and has a good feel about the different areas of the organization.\\

Make scenario for how this new work practice will be for the persons impacted by it(p191)

\subsubsection{Sharing of decisions and agreements}
....

\subsection{Technology}

\subsubsection{IT systems and IT platform}

\subsubsection{Functions}

\subsubsection{User interfaces}

\subsection{Work organisation}
\label{sub:work_organisation}

\subsection{Qualification needs}
\label{sub:qualification_needs}

\section{Advantages and disadvantages}

\section{Finances}

\section{Implementation strategy}

\texttt{PODIO APP}
\section{Visions for change}

\subsection{Technology}
\label{sub:technology}

\subsubsection{IT systems and IT platform}

\subsubsection{Functions}

\subsubsection{User interfaces}

\subsection{Work organisation}
\label{sub:work_organisation}

\subsection{Qualification needs}
\label{sub:qualification_needs}

\section{Advantages and disadvantages}


\section{Finances}

\section{Implementation strategy}


