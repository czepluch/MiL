%!TEX root = /Users/Abj/git/MiL/report/report.tex
\part{Innovation}
As the previous sections of the \texttt{In-depth phase} in part \ref{prt:in_depth_analysis},
especially in section \ref{sub:goprne}, has shown,
the organisation has room for optimisation and improvements. Based on the \ref{sub:ideas} that we
have developed and the fact that the organisation wants to continue using Podio, we have chosen two that we believe will bring the most benefits to the organisation.
We will in the following analyse both the solutions, and try to determine which will be the best or
if a third solution will emerge from this analysis. A short description of the ideas in focus
follows:


1. Work processes for Podio\\
In the previous the Diagnostic Map in section revealed several problem and related causes found in the organization. In 5b its said that the cause of this problem, is that their is too much off topic discussion and also that no real difference between pure informational topics and topics for discussions are in place within the organizational use of Podio. As Stine F also points out in 1a, the organization faces a problem about sharing of knowledge, since this is not documented properly on Podio.\\

This solution proposal will attempt to structure work processes when using Podio and make guidelines for the information posted at Podio. This proposal will therefore be directed at the use of Podio, with respect to the mentioned things in the above, and will keep to the organizations our strategy about using Podio as their primary IT solution. 

2. Podio App
This solution consists of the development of a Podio application \footnote{https://developers.podio.com/doc/app-store}.
The application will be used to create, store, and visualise tasks in- and externally in the teams.
Thus, hopefully, making it very transparent which tasks are due to when.

\texttt{WORK PROCESSES}
\section{Visions for change}

\subsection{Technology}
\label{sub:technology}

\subsubsection{IT systems and IT platform}

\subsubsection{Functions}

\subsubsection{User interfaces}

\subsection{Work organisation}
\label{sub:work_organisation}

\subsection{Qualification needs}
\label{sub:qualification_needs}

\section{Advantages and disadvantages}

\section{Finances}

\section{Implementation strategy}




\texttt{PODIO APP}
\section{Visions for change}
As mentioned earlier, the organisation wants to continue to use Podio as their platform of
communication. This means that we will have to develop a solution based on their current IT system.
As far as we are concerned the changes will not be too significant, and since a lot of the users of
the system have very little experience with Podio, they will not experience a major change in their
work practices. 

Furthermore since the need of \mil does not require a big change in the work practices, but rather
just a structured way to manage certain tasks, it is very suitable to continue using Podio. (They
are sponsored by Podio and they cannot afford a customised IT system either.)

We will now run through our vision of the solution.

\subsection{Technology}
\label{sub:technology}
Since we, as mentioned in section \ref{sub:goprne} are bound to use Podio, the technology of choice
for our solution is naturally Podio. 

Podio allows the users to create and build their own applications, to use within their workspaces.
It does however require some skills and knowledge to do this, and if the application is required to
have some complicated functionality it might be preferred to get a professional to implement it.

This solution proposition builds on such an application for Podio that, in basic terms, makes it possible to add
tasks with a given deadline, dependency, and responsible person. A visualisation of the tasks with
other dependent tasks, must also be available within the application. 

\subsubsection{IT systems and IT platform}

As mentioned above the IT platform for this IT solution will be Podio. Podio is a good fit for an
organisation as \mil since it offers some great features regarding organisations, and it makes it
very easy to implement an organisational structure of workspaces. It also allows for the users to
build their own customised applications, which makes it possible for an organisation, to use a
standardised platform with a lot of basic functionality, to get their unique requirements fulfilled.
This is what makes Podio very suitable for an organisation like \milNO, where the budget is very
tight and it would require a lot of work to move all the already existing data to another system.

\subsubsection{Functions}
In the following a list of each functionality that the system will have is shown.
\begin{itemize}
    \item Creation of a task
    \item A task will have title, description, deadline, dependent task(s), owner, and responsible
    team.
    \item It must be possible to enable notifications to be received at a desired time before the
    deadline.
    \item Teams responsible for a dependent task, will be notified when another task is depending on
    it.
    \item It must be possible to see a visualisation of all tasks, in the form of a gantt chart.
    \item The board must be notified if a task does not satisfy a deadline.
    \item It must be possible to set a task as done.
    \item When a task is done, it must be possible to save the information documenting that the
    given task is done. (Ex a signed contract for the fence surrounding the festival area.)
\end{itemize}

\subsubsection{User interfaces}
Podio has its own user interface, so there are not many changes to make in this area. The members of
the organisation have been using Podio for some time, which makes them somewhat familiar with the
user interface of Podio. We see this as an advantage, since we do not need to spend too many
resources on making the users of Podio used to the Podio user interface.

\subsection{Work organisation}
\label{sub:work_organisation}
The adjustments to the work organisation is very limited in this case. There are however some
minor adjustments that are required for the solution to be implemented successfully. 
Most importantly is it, to properly educate all the members of the Podio organisation, in the use of
the application, thus making sure everyone knows that the application exists and how to use it. It
is especially important for the team leaders to know how the application works, since they will be
the ones who mainly creates tasks. 

\subsection{Qualification needs}
\label{sub:qualification_needs}

\section{Advantages and disadvantages}


\section{Finances}

\section{Implementation strategy}


