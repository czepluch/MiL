\section{Interview 4: 20.11.13.Andreas}
\label{sec:interview_4}

\textbf{Dato:} 21.11.2013 \\
\textbf{Setting:} IT Universitetet, København \\
\textbf{Musik i Lejet repræsentanter:} Bestyrelsesmedlem og arrangør, Andreas Grauengaard  \\
\textbf{Projekt repræsentanter:} Jakob Helvind, Sune Debel, Jacob Stenum Czepluch

\bigskip

\noindent \textbf{Summary} \\
<<<<<<< HEAD
Detter interview blev foretaget på IT Universitetet d. 21. november med Andreas Grauengaard. Vi ønskede at afholde mødet, da vi havde en del opklarende spørgsmål vedrørende en række forskellige ting, samt for at bekræfte en række af de observationer og informationer vi havde erhvervet os i forløbet indtil nu. En del af spørgsmålene var omhandlede \mil rent organisatorisk, men vi fandt senere ud af, at disse spørgsmål ingen relevans havde for vores løsning og opgave. Derfor har vi udeladt disse spørgsmål, da vi alligevel ikke bruger dem til noget. De spørgsmål med relevans for
=======
Dette møde blev afholdt, da vi havde en del opklarende spørgsmål vedrørende en række forskellige
ting, samt for at bekræfte en række af de observationer og informationer vi havde erhvervet os i
forløbet indtil nu. En del af spørgsmålene var omhandlede \mil rent organisatorisk, men vi fandt
senere ud af, at disse spørgsmål ingen relevans havde for vores løsning og opgave. Derfor har vi
udeladt disse spørgsmål, da vi alligevel ikke bruger dem til noget. De spørgsmål med relevans for
>>>>>>> b9f0dce1c7217a682187c3e3141b9683f7193fac
vores opgave følger herunder.

\subsection{Question 1}
\label{i3q1}
\noindent \textbf{Hvilke økonomiske konsekvenser har omstruktureringen af organisationen haft?} 
\\
Den største konsekvens er nok, at der i og med, at der er blevet lavet teams, også er blevet lavet
et seperat budget til hvert ansvarsområde. Det er jeg personligt lidt ked af, da jeg tidligere har
lagt stor vægt på og energi i, at vi altid fandt de bedste tilbud til prisen, men jeg frygter nu, at
dette ikke altid kommer til at være tilfældet. Grunden til dette er, at jeg der ikke er nogen
kontrol med om tilbuddene er de bedste, så længe budgettet bare bliver overholdt i hver individuelt
team.

\subsection{Question 2}
\label{i3q2}
\noindent \textbf{Har du tidligere oplevet, at I ikke fik det bedst mulige tilbud?}
\\
Ja, vi oplevede sidste år, at vi måtte betale 10.000 mere for hegnet, end der var budgetteret med,
fordi den ansvarlige for at bestille hegn var alt for sent ude med at indhente tilbud på hegn, og
måtte derfor vælge det første og det bedste, da der ikke var tid til at undersøge alternativer til
en bedre pris.

\subsection{Question 3}
\label{i3q4}
\noindent \textbf{Hvor ofte vil du vurdere at sådan en situation opstår?}
\\
Det er svært at svare på, men et slag på tasken vil være et par gange på et år.

\subsection{Question 4}
\label{i3q5}
\noindent \textbf{Det vil sige at risikoen ved at udskyde at indhente tilbud er ret stor?}
\\
Ja det vil jeg sige, for jo nærmere vi kommer festivalen, jo sværrere bliver det at forhandle prisen.

\subsection{Question 4}
\label{i3q5}
\noindent \textbf{Kan du give et overslag på hvad det koster jer når sådan en situation opstår?}
\\
Det er svært at sige, men det er vel i omegnen af et par tusinde.

\subsection{Question 5}
\label{i3q3}
\noindent \textbf{Hvilke andre lignende situationer har der været?}
\\
Vi oplevede også sidste år, at de personer, der var ansvarlige for selve festivalpladsen, havde
ønsket ekstra toiletter til pladsen. Dette havde vi dog ikke opfattet i bestyrelsen, da de havde
skrevet det på Podio, som vi lige var begyndt, at bruge, og ingen af os fra bestyrelsen havde fået
noteret det ned noget sted, og det forsvandt derefter bare i strømmen. Da festivalen så nærmede sig
og personerne gav os regningen for de ekstra toiletter, gik vi over budget, da dette ikke var blevet
regnet med i budgettet.
