\section{Interview 2: 31/10/2013}
\label{interview_2}

\textbf{Dato:} 31.10.2013 \\
\textbf{Setting:} Getinge offices, Copenhagen \\
\textbf{Musik i Lejet repræsentanter:} Bestyrelsesmedlemmer Kristian Graengaard og Søren Stakemann \\
\textbf{Project repræsentanter:} Jacob Stenum Czepluch, Daniel Varab, Sune Debel.

\bigskip

\noindent \textbf{Summary} \\
Dette interview blev foretaget den 31. oktober i et mødelokale hos virksomheden Getinge hvoraf 2 af 4 af de nuværende bestyrelsesmedlemmer var til stede. Mødelokalet var til rådighed gennem Søren Stakemann da han til dagligt arbejder i koncernen, og anvendes allerede af Musik i Lejet til bestyrelsesmøder. Fokus ved dette møde var at få yderligere indsigt i hvad Musik i Lejet som organisation indeholdt såvel som at prikke hul på hvilke udfordringer der forekom.


\subsection{Question 1}
\label{i2q1}
\noindent \textbf{Hvad er Musik i Lejets formål som organisation?} \\
Vores formål er at præsentere den mest ambitiøse og interessante musik fra den danske musikscene. Vi ønsker også at skabe kultur i Tisvildeleje gennem dansk musik. Vores eneste økonomiske målsætning er at være i stand til at afholde næste års festival, da vi er en non-profit organisation. Vi giver ikke noget til velgørenhed. 

\subsection{Question 2}
\label{i2q2}
\noindent \textbf{Har I selv nogle tanker om, hvad det er for nogle grupper der kommer på MiL?} \\
Vi har ikke en konkret defineret målgruppe, men rettere en  meget stram musikprofil, der tiltrækker nogle bestemt grupper. Vi fokuserer dog på mangfoldigheden i og med, at vi har gratis entré for folk under 12 og over 60, så der kommer også en del børnefamilier og ældre. Derfor har vi også en børnescene. Vores musikprofil og vores mangfoldighed er noget det jeg synes adskiller os fra andre festivaler. Det er helt sikkert også noget af det vi skal konkurrere på. Vi er jo også meget i dialog med de andre festivaler, bl.a. igennem DanskLive \footnote{http://dansklive.dk/}

\subsection{Question 3}
\label{i2q3}
\noindent \textbf{Hvad er det så for nogle folk der kommer grundet musikprofilen?} \\
Man kan vel dele dem op i to hovedgrupper. Dem der kommer for musikken før midnat er dem vi kalder de 28-årige og de er vores hovedmålgruppe. 

\subsection{Question 4}
\label{i2q4}
\noindent \textbf{Kan du uddybe de 28-årige?} \\
Det er et mere modent publikum end dem man ser på Roskilde festivalen blandt andet og vi vil gerne selv kunne sige, at de er relativt sofistikerede. Det er også dem der kommer for den unikke “hjemmelavede” stemning som vi mener vi har. Og så er det faktisk ret interessant at der 65\% piger på festivalen. Derudover havde billetten i år en pris, der tillod folk at komme koncertbevidst, hvilket resulterede i, at der var nogle der først indløste deres billet den sidste dag, for at høre en specifik koncert. Det gjorde også at folk der kom for musikken købte deres billet i god tid.

\subsection{Question 5}
\label{i2q5}
\noindent \textbf{Hvad er den anden gruppe?} \\
Den anden gruppe er dem der kommer efter midnat til nattefesten. Til nattefesten bliver der spillet andet musik end om dagen og det er hovedsageligt folk der kommer for at feste, der kommer her. Det er disse fester der tjener pengene ind, der gør det muligt at afholde næste års festival, så vi kan ikke klare os uden dem. De trækker selvfølgelig festivalen i en lidt anden retning end børnefamilier, og de stiller også et væsentligt størrere krav til baren.

\subsection{Question 6}
\label{i2q6}
\noindent \textbf{Gør I noget for at inddrage lokalbefolkningen?} \\
Vi synes det er vigtigt at have lokalbefolkningen med og vi inddrager også den lokale skole til oprydningen, så de kan få nogle penge til deres klassekasser. Og så betyder det selvfølgelig meget for os, at lokalsamfundet, synes det er en god idé, at vi afholder festivalen. Vi vil jo gerne præge Tisvilde og dens fremtid. Vi er også med i Tisvildegruppen som er en lokal gruppe af foreninger der er sammen om at forbedre lokalsamfundet. Der er jo også en del vi samarbejder med, fx lokale restauranter der har en madbod på festivalen, eller at vi har plakater og flyers i deres butikker.

\subsection{Question 7}
\label{i2q7}
\noindent \textbf{Hvad kommer folk på Musik i Lejet for?} \\
Som sagt den unikke og stramme udelukkende danske musikprofil. Derudover har vores festival en helt unik beliggenhed - den ligger jo bogstaveligt talt på stranden. Vi mener selv at vores publikum er meget entusiastisk hvilket skaber en helt bestemt stemning for musikelskere. Tisvilde er lidt snobbet i sommerferieperioden, og det prøver vi at nedtone lidt ved at have en meget afslappet stemning. Vi prøver at nedtone kontrasten mellem de “rige” og “hippierne”.

\subsection{Question 8}
\label{i2q8}
\noindent \textbf{Vi ved fra mødet med Andreas, at I laver en organisationsomlægning. Hvorfor?} \\
Festivalen er blevet større og hver person i bestyrelsen havde rigtig meget ansvar og meget viden, men dette ansvar og viden var ikke delt. Der var ingen der havde hele overblikket, hvilket betød, at der var mange uundværlige personer. Vi, i bestyrelsen, ønskede at kunne fralægge os noget af det enorme ansvar, samt kunne få taget beslutninger hurtigere, og i sidste ende fokusere på strategiske og økonomiske beslutninger for hele organisationen. 

\subsection{Question 9}
\label{i2q9}
\noindent \textbf{Hvad består omlægningen i?} \\
Den består hovedsageligt i at flytte mange beslutninger væk fra bestyrelsens ansvarsområde. Vi vil derfor lave ledelsesgrupper, der har hver sit overordnede ansvarsområde og som er i stand til at tage beslutninger herom, som mødes hver 6 uge. Store beslutninger skal dog stadig vendes med bestyrelsen og vi vil også i bestyrelsen selv stadig stå for nogle af de helt vitale ansvarsområder. 

\subsection{Question 10}
\label{i2q10}
\noindent \textbf{Hvordan vil I undgå, at alle beslutninger fortsat bliver vendt med jer?} \\
Vi har ikke helt fundet en god løsning på dette endnu, men det er noget af det, vi skal diskutere på vores første ledelsesmøde den 11. november samt på vores workshop weekend den 16 og 17. november. Men der er ingen tvivl om, at vi skal gøre noget aktivt for at fremme kommunikationen mellem teamlederne for at undgå at alt vendes med os i bestyrelsen.

\subsection{Question 11}
\label{i2q11}
\noindent \textbf{Hvilke problemstillinger mener I, at der er ved at indføre disse teamledere?} \\
Det er kan være svært at fralægge sig ansvaret for det område, man har været vant til at stå for i mange år. Det kræver en stor omstilling fra vores side i bestyrelsen at få det gennemført. Derudover skal vi blive bedre til at bruge Podio, som er vores kommunikationsmedie. Der er alt for meget information der drukner i mængden, da vi lidt for ofte lader kommentarene køre lidt ud på et sidespor. Det er ikke meningen, at der skal foregå nogen kommunikation uden for Podio, men det gør der desværre. Dette vil vi gerne lave om på og forbedre.