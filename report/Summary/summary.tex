%!TEX root = /Users/Abj/git/MiL/report/report.tex
\part{Summary}

\section{About}
This paper is a business case proposition for the board of Musik i Lejet, a Danish music festival held each year at Tisvildeleje. In cooperation with the board and other arrangers, we have investigated MiL (Musik i Lejet) as an organisation, looking at its structure and work practices. This business case will try to uncover areas of interest, where an it-solution or other change, might help lessen problems and barriers, as to provide better value for MiL and its customers.

Musik i Lejet is a festival planned and executed by a group of volunteer arrangers. Over the last couple of years the festival has grown, and so has the workload and number of arrangers. The work tasks  are split up, and delegated to different work-areas, but are still dependent upon each other, delays in one postponing the other. As the whole planning phase has a ultimo deadline each year, the festival itself, tasks can only be pushed so far. Therefore we have chosen to focus our investigation around the following:\\

How can the deadlines missed by team leaders, agreed upon at team leader meetings, be reduced by xx percentage.

\section{List of activities}
The following is a list of different activities we have conducted, consisting of interviews with and observations of the members of Mil.
\begin{center}
\begin{table}
    \begin{tabular}{|p{3cm}|p{3cm}|p{3cm}|p{6cm}|}
    \hline
    Date & Type of activity & Participants & Comments \\
    29 - 09 - 13 & (informal) Interview & MiL: Andreas, ITU: All & The workgroup presented the scope(?) of the course and got an initial idea of the organization and the role that Andreas have in MiL.  \\
    31 - 10 - 13 & (formal) Interview & MiL: Christian og Stakkeman, ITU: Daniel, Jacob, Sune &  .......  \\
    17 - 11 - 13 & Observations & MiL: Board and Arrangers, ITU: Anders, Daniel, Jacob & Combined workshop and general assembly. \\
    20 - 11 - 13 & (formal) Interview & MiL: Andreas, ITU: Jacob, Jakob, Sune & .... \\
    21 - 11 - 13 & (formal) Interview & MiL: Stine, ITU: Jakob, Sune & .... \\
    \end{tabular}
\end{table}
\end{center}

\section{Figure of organization}
\label{sec:organisation}
This is a section of an organisational chart outlining the structure of Mil, made in collaboration with the board. The whole chart can be found in appendix XX: \\
\includegraphics[scale=0.5]{Pictures/MIL_Organisational_chart_Cutout.jpg}
\section{Business and IT scope}
In this investigation we will focus on the managing part and more specific the arranger teams and team leaders of Musik I lejet. We will also look at how the internal communication is done, how information is shared between all members and what work processes are in play when planning the festival. \\
Most of the communication is done at Musik i Lejet's Podio site and therefore we will direct our attention and observations towards that platform. This is also the main IT-system used in the planning phase of the festival, when this is done, and as such this is where our focus on IT will be placed.
